\documentclass{beamer}

\usepackage[T1]{fontenc}
\usepackage[utf8]{inputenc}

%Information to be included in the title page:
\title[About Beamer] %optional
{About the Beamer class in presentation making}

\subtitle{A short story}

\author[Arthur, Doe] % (optional, for multiple authors)
{A.~B.~Arthur\inst{1} \and J.~Doe\inst{2}}

\institute[VFU] % (optional)
{
  \inst{1}%
  Faculty of Physics\\
  Very Famous University
  \and
  \inst{2}%
  Faculty of Chemistry\\
  Very Famous University
}

\date[VLC 2013] % (optional)
{Very Large Conference, April 2013}

% \usefonttheme{structuresmallcapsserif} % themes are: structurebold, structurebolditalic, structuresmallcapsserif, structureitalicsserif, serif and default
\usepackage{bookman} % mathptmx, helvet, avat, bookman, chancery, charter, culer, mathtime, mathptm, newcent, palatino, pifont and utopia
\usetheme{Madrid}
\usecolortheme{beaver}

\begin{document}

  \frame{\titlepage}

  \section{Section 1}
    \subsection{sub a}

      \begin{frame}
        \frametitle{Outline}
        \tableofcontents
      \end{frame}

      \begin{frame}
        \frametitle{Sample frame title}
        This is a text in the first frame. This is a text in the first frame. This is a text in the first frame.
      \end{frame}

      \begin{frame}
        \frametitle{List}
        \begin{itemize}
          \item Point A
          \item Point B
            \begin{itemize}
              \item part 1
              \item part 2
            \end{itemize}
          \item Point C
          \item Point D
        \end{itemize}
      \end{frame}

      \begin{frame}
        \begin{enumerate}[I]
          \item Point A
          \item Point B
            \begin{enumerate}[i]
              \item part 1
              \item part 2
            \end{enumerate}
          \item Point C
          \item Point D
          \end{enumerate}
      \end{frame}

      \begin{frame}
        \frametitle{Sample frame title}
        This is a text in second frame.
        For the sake of showing an example.

        \begin{itemize}
        \item<1-> Text visible on slide 1
        \item<2-> Text visible on slide 2
        \item<3> Text visible on slide 3
        \item<4-> Text visible on slide 4
        \end{itemize}

      \end{frame}

      \begin{frame}
        In this slide \pause

        the text will be partially visible \pause

        And finally everything will be there
      \end{frame}

      \begin{frame}
        \frametitle{Sample frame title}

        In this slide, some important text will be
        \alert{highlighted} because it's important.
        Please, don't abuse it.

        \begin{block}{Remark}
          Sample text
        \end{block}

        \begin{alertblock}{Important theorem}
          Sample text in red box
        \end{alertblock}

        \begin{examples}
          Sample text in green box. The title of the block is ``Examples".
        \end{examples}
      \end{frame}

      \begin{frame}
        \frametitle{Two-column slide}

        \begin{columns}
          \column{0.5\textwidth}
            This is a text in first column.
            $$E=mc^2$$
            \begin{itemize}
              \item First item
              \item Second item
            \end{itemize}

          \column{0.5\textwidth}
            This text will be in the second column
            and on a second tought this is a nice looking
            layout in some cases.
        \end{columns}
      \end{frame}

      \begin{frame}
        \begin{description}
          \item[API] Application Programming Interface
          \item[LAN] Local Area Network
          \item[ASCII] American Standard Code for Information Interchange
          \end{description}
      \end{frame}

      \begin{frame}
        \begin{table}
          \begin{tabular}{l | c | c | c | c }
            Competitor Name & Swim & Cycle & Run & Total \\
            \hline \hline
            John T & 13:04 & 24:15 & 18:34 & 55:53 \\
            Norman P & 8:00 & 22:45 & 23:02 & 53:47\\
            Alex K & 14:00 & 28:00 & n/a & n/a\\
            Sarah H & 9:22 & 21:10 & 24:03 & 54:35
          \end{tabular}
          \caption{Triathlon results}
        \end{table}
      \end{frame}

      \begin{frame}
        \begin{block}{Block Title}
          Lorem ipsum dolor sit amet, consectetur adipisicing elit,
          sed do eiusmod tempor incididunt ut labore et
          dolore magna aliqua.
        \end{block}
        \begin{alertblock}{Block Title}
          Lorem ipsum dolor sit amet, consectetur adipisicing elit,
          sed do eiusmod tempor incididunt ut labore et
          dolore magna aliqua.
        \end{alertblock}
        \begin{definition}
          A prime number is a number that...
        \end{definition}
        \begin{example}
          Lorem ipsum dolor sit amet, consectetur adipisicing elit,
          sed do eiusmod tempor incididunt ut labore et
          dolore magna aliqua.
        \end{example}
      \end{frame}

      \begin{frame}
        \begin{theorem}[Pythagoras]
          $ a^2 + b^2 = c^2$
        \end{theorem}
        \begin{corollary}
          $ x + y = y + x  $
        \end{corollary}
        \begin{proof}
          $\omega +\phi = \epsilon $
        \end{proof}
      \end{frame}

      \begin{frame}[fragile]
        \frametitle{Including Code}
        \begin{semiverbatim}
          \\begin\{frame\}
          \\frametitle\{Outline\}
          \\tableofcontents
          \\end\{frame\}
        \end{semiverbatim}
      \end{frame}
\end{document}
