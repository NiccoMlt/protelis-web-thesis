\section{WebProtelis}

    \subsection{Progettazione del server}
      \begin{frame}{\insertsectionhead}
        \framesubtitle{\insertsubsectionhead}

        \begin{block}{Servizi offerti}
          Il backend è composto da due entità principali
          \begin{itemize}
            \item server che espone API per la comunicazione remota
            \item esecutore del codice Protelis
          \end{itemize}
        \end{block}

        \begin{block}{Scelte tecnologiche}
          \begin{itemize}
            \item Vert.x      % TODO: spiega meglio
            \item Kotlin      % TODO: spiega meglio
            \item Alchemist   % TODO: spiega meglio
          \end{itemize}
        \end{block}
      \end{frame}

      \begin{frame}{\insertsectionhead}
        \framesubtitle{\insertsubsectionhead}

        \begin{block}{Vert.x}
          \emph{Vert.x} è un framework applicativo event-driven che esegue su JVM.
          Del modello architetturale messo a disposizione dal framework, è stato considerato interessante il concetto di \emph{Verticle}:
          esso è un'astrazione, simile al pattern ad attori ma non considerato pienamente aderente al modello teorico dalla stessa documentazione ufficiale,
          che incapsula un event-loop insieme al suo stato e interagisce tramite gli eventi provenienti da un EventBus.
        \end{block}

      % \begin{frame}{\insertsectionhead}
      %   \framesubtitle{\insertsubsectionhead}

        \begin{block}{Pattern reactor}
          % TODO: spiega meglio
        \end{block}
      \end{frame}

      \begin{frame}{\insertsectionhead}
        \framesubtitle{\insertsubsectionhead}

        \begin{block}{Alchemist}
          % TODO: spiega meglio
        \end{block}
      \end{frame}

    \subsection{Progettazione del client}

      \begin{frame}{\insertsectionhead}
        \framesubtitle{\insertsubsectionhead}
        \begin{block}{Scelte tecnologiche}
          \begin{itemize}
            \item React   % TODO: spiega meglio
            \item Sock.js % TODO: spiega meglio
          \end{itemize}
        \end{block}
      \end{frame}

    \subsection{Valutazione dei risultati}

