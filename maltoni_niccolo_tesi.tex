% !TeX root = ./maltoni_niccolo_tesi.tex
% !TeX encoding = UTF-8 Unicode
% !TeX spellcheck = it_IT
% !TeX program = arara
% !TeX options = --log --verbose --language=it "%DOC%"

% arara: lualatex:      { interaction: batchmode }
% arara: frontespizio:  { interaction: batchmode, engine: lualatex }
% arara: biber
% arara: lualatex:      { interaction: batchmode }
% arara: lualatex:      { interaction: nonstopmode, synctex: yes }

\documentclass[%
  a4paper,                % formato di pagina A4
  12pt,                   % corpo del testo a 12pt
  % la dimensione 12pt automaticamente imposta \footnotesize a 10pt
  twoside,                % (oneside|twoside) documento a singola o doppia facciata,
  openright,              % (openany|openright) fa cominciare un capitolo nella successiva pagina a disposizione o sempre in una pagina destra
  % twocolumn,            % dà a LaTeX le istruzioni per comporre l'intero documento su due colonne
  titlepage,              % (titlepage|notitlepage) se dopo il titolo del documento debbaavere  inizio  una  nuova  pagina
  % fleqn,                % allinea le formule a sinistra rispetto a un margine rientrato
  % leqno,                % mette la numerazione delle formule a sinistra anziché a destra
  final                   % (draft|final) scelta tra bozza o finale, influenza il comportamento degli altri pacchetti
]{scrbook}

\usepackage{fancyvrb}       % fornisce l'ambiente VerbatimOut e modifica listati di codice

\begin{VerbatimOut}{\jobname.xmpdata}
\Title{Progettazione di un sistema web per la simulazione di programmi aggregati}
\Author{Niccolò Maltoni}
\Copyright{Questo documento è fornito sotto licenza Creative Commons Attribution-ShareAlike 4.0 International}
\CopyrightURL{http://creativecommons.org/licenses/by-sa/4.0}
\end{VerbatimOut}

\usepackage{unibotesi}

\begin{document}

  \frontmatter{}
  \pagenumbering{Roman}
  \pagestyle{empty}
  \begin{Preambolo*}
  \usepackage{fontspec}
  \defaultfontfeatures{ Scale = MatchUppercase }
  \setmainfont{libertinusserif}[
    Scale=1.0,
    Ligatures={Common, TeX},
    % Numbers={OldStyle, Proportional},
    UprightFont={*-regular},
    BoldFont={*-bold},
    ItalicFont={*-italic},
    BoldItalicFont={*-bolditalic},
    Extension=.otf
  ]
  \setsansfont{libertinussans}[
    Ligatures={Common, TeX},
    % Numbers={OldStyle, Proportional},
    UprightFont={*-regular},
    BoldFont={*-bold},
    ItalicFont={*-italic},
    % BoldItalicFont={*-bolditalic},
    Extension=.otf
  ]
  \setmonofont{libertinusmono}[
    Scale=0.95,
    UprightFont={*-regular},
    % BoldFont={*-bold},
    % ItalicFont={*-italic},
    % BoldItalicFont={*-bolditalic},
    Extension=.otf
  ]
\end{Preambolo*}
\begin{frontespizio}
  \Universita{Bologna}        % aggiunge da sé “Università degli Studi di”.
  \Istituzione{%
    Alma Mater Studiorum --- Università di Bologna \\%
    Campus di Cesena%
  }
  \Divisione{Dipartimento di Informatica --- Scienza e Ingegneria}
  \Corso[Laurea magistrale]{Ingegneria e Scienze Informatiche}
  \Annoaccademico{2019--2020}
  \Titolo{Progettazione di una piattaforma web\\per la simulazione di programmi aggregati}
  \Sottotitolo{Tesi in Pervasive Computing}
  % \Preambolo{\renewcommand{\frontsmallfont}[1]{\small}}       % non viene stampata la matricola
  % \Preambolo{\renewcommand{\frontsmallfont}[1]{\small Matr.}} % abbrevia la matricola
  \Candidato[840825]{Niccolò~Maltoni}
  \NCandidato{Presentata da}  % sostituisce la parola “Candidato”
  \Relatore{Prof.~Mirko~Viroli}
  \Correlatore{Prof.~Danilo~Pianini}
  \Piede{%                    % sostituisce la scritta “Anno Accademico” nel piede
    III sessione di laurea \\%
    Anno Accademico 2019--2020%
  }
\end{frontespizio}

% Necessario per Overleaf: compila il TeX del frontespizio subito dopo averlo generato
\IfFileExists{\jobname-frn.pdf}{}{%
\immediate\write18{lualatex \jobname-frn}}

  \clearemptydoublepage{}
\thispagestyle{empty}
\vspace*{20ex}
\begin{flushright}
    \begin{LARGE}
        \textbf{Parole chiave}\\
        \vspace{5ex}
    \end{LARGE}
    \begin{normalsize}
        \textbf{%
            Parola chiave 1\\%
            \medskip
            Parola chiave 2%
        }
    \end{normalsize}
\end{flushright}
\vfill

  % !TeX root = ../../maltoni_niccolo_tesi.tex
% !TeX encoding = UTF-8 Unicode
% !TeX spellcheck = it_IT

\clearemptydoublepage{}
\null{}\vspace{\stretch{1}}
\begin{flushright}
    \textit{Dedica}
\end{flushright}
\vspace{\stretch{2}}\null{}

  \begin{abstract}
  % \strong{TODO}
  \todo[inline]{Il sommario verrà scritto per ultimo; di seguito un lorem ipsum per dare l'idea}
  \lipsum[1-2] % ChkTeX 8
\end{abstract}

  \tableofcontents

  \mainmatter{}
  \pagenumbering{arabic}
  \pagestyle{headings}

  \begin{abstract}
    % Abstract
  \end{abstract}

  \chapter*{Introduzione}

  \part{Background}
    \chapter{Motivazioni}
      \section{Stato dell'arte}
      \section{Prospettive}
      \section{Dettaglio del problema e approccio}

    \chapter{Programmazione aggregata}
      \section{Field calculus}
      \section{Protelis}
      \section{Scafi}

    \chapter{Progettazione di sistemi web}
      \section{Architetture, framework e stack}
      \section{Linguaggi ad uso web}
        \subsection{JavaScript e ECMAScript}
        \subsection{TypeScript}
        \subsection{Scala.js}
        \subsection{Kotlin/Multiplatform e Kotlin/JS}

  \part{Contributo}
    \chapter{Protelis-Web}
      \section{Analisi dei requisiti}
        \subsection{Requisiti funzionali}
        \subsection{Requisiti non funzionali}
      \section{Progettazione}
        \subsection{Design dell'architettura}
        \subsection{Mockup dell'interfaccia}
      \section{Implementazione}

  \part{Conclusioni}
    \chapter{Valutazione dei risultati}
    \chapter{Lavori futuri}
    \chapter{Considerazioni finali}

  \appendix
  \addpart*{\appendixname}
% \renewcommand{\thesection}{\Alph{section}}
% \renewcommand{\thesubsection}{A.\arabic{subsection}}

% \addcontentsline{toc}{section}{\appendixname}
% \addchap{\appendixname}
% \chapter*[]{\appendixname}

\begin{appendices}
  % \section{Dockerfile del server}\label{app:docker}
  \chapter{Dockerfile del server}\label{app:docker}
    \inputminted{dockerfile}{res/code/Dockerfile}
\end{appendices}


  \backmatter{}
  % \nocite{7274429,PianiniSASOTutorial2017}
\printbibliography[heading=bibintoc]

  \addchap{Ringraziamenti}

% 1
% Ringrazio il professor Mirko Viroli e il professor Danilo Pianini per la bella
% opportunità che mi hanno offerto, per tutto l’aiuto datomi per realizzare
% questo progetto e per i consigli ricevuti. Ringrazio anche gli amici che mi
% sono stati accanto durante questi mesi, ma soprattutto un particolare grazie
% ai miei genitori che mi sostengono da sempre.

% 2
% Un sincero ringraziamento va a tutti coloro che mi hanno aiutato in vario modo a
% raggiungere questo traguardo. Ringrazio il professor Mirko Viroli e il professor Danilo
% Pianini per l’aiuto datomi nella realizzazione di questo progetto, sia durante l’attività
% sperimentale che per la stesura dell’elaborato finale. Ringrazio anche gli amici che ho
% avuto vicino in questi mesi, ma soprattutto un particolare grazie ai miei familiari che mi
% sostengono da sempre.


\end{document}
