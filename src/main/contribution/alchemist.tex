\subsection{Simulatore scelto: Alchemist}\label{subsec:alchemist}
  Alchemist~\cite{alchemist-jos2013} è un meta-simulatore estendibile completamente \emph{open-source} che esegue su \engEmph{Java~Virtual~Machine}, nato all'interno dell'Università di Bologna.

  \subsubsection{Simulazione}\label{subsec:introAlchemist}
    In generale, una \emph{simulazione}~\cite{des3} è una riproduzione del modo di operare di un sistema o un processo del mondo reale nel tempo.
    L'imitazione del processo del mondo reale è detta \emph{modello};
    esso risulta essere una riproduzione più o meno semplificata del mondo reale, che viene aggiornata ad ogni passo di esecuzione della simulazione.

    Alchemist rientra nell'archetipo dei simulatori ad eventi discreti (DES)~\cite{des, des2}:
    gli eventi sono strettamente ordinati e vengono eseguiti uno alla volta, determinando il passare del tempo.
    L'idea dietro al progetto è quello di riuscire ad avere un framework di simulazione il più possibile generico, in grado di simulare sistemi di tipologia e complessità diverse, mantenendo le prestazioni dei simulatori non generici (come ad esempio quelli impiegati in ambito chimico~\cite{gillespie1976}).

    Per perseguire questo obiettivo, la progettazione dell'algoritmo è partita dallo studio del lavoro di Gillespie del 1977~\cite{gillespie1977} e di altri scienziati nell'ambito della simulazione chimica.
    Nonostante siano presenti algoritmi in grado di eseguire un numero di reazioni addirittura in tempo costante, la scelta dell'algoritmo è infine ricaduta su una versione migliorata dell'algoritmo SSA di Gillespie, il Next Reaction Method~\cite{nextReactionMethod} di Gibson e Bruck:
    ad ogni passo di simulazione, esso è in grado di selezionare la reazione successiva in tempo costante e richiede un tempo logaritmico per aggiornare le strutture dati interne al termine dell'esecuzione dell'evento.

  \subsubsection{Astrazioni e modello}\label{subsec:modello}

    Il modello di astrazione di Alchemist è ispirato dal lavoro della comunità scientifica nell'ambito dei simulatori a scopo di ricerca chimica e ne riprende dunque la nomenclatura.
    Le entità (visibili in \Cref{fig:alchemist:model}) su cui lavora sono le seguenti:

    \begin{description}
      \item[Molecola]\label{itm:mol}
        Una \emph{molecola} rappresenta il nome assegnato ad un particolare dato all'interno di un \emph{nodo}, del quale ne astrae parte dello stato.

        Un parallelismo con la programmazione imperativa vedrebbe la \emph{molecola} come un'astrazione del nome di una variabile.

      \item[Concentrazione]\label{itm:conc}
        La \emph{concentrazione} di una \emph{molecola} è il valore associato alla proprietà rappresentata dalla \emph{molecola}.

        Mantenendo il parallelismo con la programmazione imperativa, la \emph{concentrazione} rappresenterebbe il valore della variabile.

        \begin{figure}[tbp] % h rimosso per posizionare correttamente la footnote
          \centering
          \includegraphics[width=.85\textwidth]{res/fig/alchemist_model.png}
          \caption[%
            Rappresentazione grafica delle diverse entità di Alchemist.
          ]{%
            Rappresentazione grafica delle diverse entità di Alchemist.\\
            Figura ripresa dal sito ufficiale\protect\footnotemark.
          }%
          \label{fig:alchemist:model}
        \end{figure}
        \footnotetext{\url{http://alchemistsimulator.github.io}}

      \item[Nodo]\label{itm:node}
        Il \emph{nodo} è un contenitore di \emph{molecole} e \emph{reazioni} che risiede all'interno di un \emph{ambiente} e che astrae una singola entità.

      \item[Ambiente]\label{itm:env}
        L'\emph{ambiente} è l'astrazione che rappresenta lo spazio nella simulazione ed è l'entità che contiene i \emph{nodi}.

        Esso è in grado di fornire informazioni in merito alla posizione dei \emph{nodi} nello spazio, alla distanza tra loro e al loro vicinato;
        opzionalmente, l'\emph{ambiente} può offrire il supporto allo spostamento dei \emph{nodi}.

      \item[Regola di collegamento]\label{itm:linkr}
        La \emph{regola di collegamento} è una funzione dello stato dell'\emph{ambiente} che associa ad ogni \emph{nodo} un \emph{vicinato}.

      \item[Vicinato]\label{itm:neigh}
        Un \emph{vicinato} è un'entità costituita da un \emph{nodo} detto ``centro'' e da un insieme di altri \emph{nodi} (i ``vicini'').

        L'astrazione dovrebbe avere un'accezione il più possibile generale e flessibile, in modo da poter modellare qualsiasi tipo di legame di vicinato, non solo spaziale.

        \begin{figure}[htbp]
          \centering
          \includegraphics[width=.85\textwidth]{res/fig/alchemist_reaction.png}
          \caption{Rappresentazione grafica della \emph{Reazione}.}%
          \label{fig:alchemist:reaction}
        \end{figure}

      \item[Reazione]\label{itm:react}
        Il concetto di \emph{reazione} è da considerarsi molto più elaborato di quello utilizzato in chimica:
        in questo caso, si può considerare come un insieme di \emph{condizioni} sullo stato del sistema, che qualora dovessero risultare vere innescherebbero l'esecuzione di un insieme di \emph{azioni}.

        Una \emph{reazione} (di cui si ha una rappresentazione grafica in~\Cref{fig:alchemist:reaction}) è dunque un qualsiasi evento che può cambiare lo stato dell'\emph{ambiente} e si compone di un insieme di \emph{condizioni}, una o più \emph{azioni} e una distribuzione temporale.

        La frequenza di accadimento può dipendere da:
        \begin{itemize}
            \item Un tasso statico;
            \item Il valore di ciascuna \emph{condizione};
            \item Una equazione che combina il tasso statico e il valore delle \emph{condizioni}, restituendo un ``tasso istantaneo'';
            \item Una distribuzione temporale.
        \end{itemize}

        Ogni \emph{nodo} è costituito da un insieme (anche vuoto) di \emph{reazioni}.

      \item[Condizione]\label{itm:cond}
        Una \emph{condizione} è una funzione che associa un valore numerico e un valore booleano allo stato corrente di un \emph{ambiente}.

      \item[Azione]\label{itm:act}
        Un'\emph{azione} è una procedura che provoca una modifica allo stato dell'\emph{ambiente}.

    \end{description}

    Per quanto la terminologia sia ripresa dalla chimica, il meta-modello del simulatore è estendibile, adottando interpretazioni più o meno lasche dei termini ``molecola'' e ``concentrazione''.
    In particolare, in Alchemist esiste il concetto di \emph{incarnazione}, che definisce l'istanza concreta del meta-modello, delineando le modalità con le quali le astrazioni vengono implementate.

  \subsubsection{Incarnazione Protelis}

    Alchemist fornisce l'implementazione di diverse incarnazioni;
    per lo scopo di questa tesi, ci si propone di utilizzare l'incarnazione Protelis.
    In essa, la molecola identifica il nome di un sensore, mentre la sua concentrazione è il valore misurato.

    Attraverso la configurazione di Alchemist, è possibile definire il posizionamento dei nodi e le modalità di collegamento, nonché la presenza di specifiche molecole.
    In questo modo, è possibile definire una molecola che conterrà il codice Protelis che ciascun nodo deve eseguire;
    il sistema può così caricare dinamicamente il codice ottenendo la relativa concentrazione.
    Un esempio di configurazione è riportato in~\Cref{app:yaml}.
