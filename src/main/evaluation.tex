\chapter{Valutazione dei risultati}\label{ch:evaluation}

  \section{Qualità del codice \& testing}
    Durante lo sviluppo, si è cercato di prestare particolare attenzione alla qualità del codice prodotto.

    Si è ritenuto fondamentale, innanzitutto, per favorire la consistenza dello stile, che il codice fosse conforme a uno stile di programmazione riconosciuto.
    Come detto nella~\Cref{subec:quality}, sono stati utilizzati diversi \emph{linter} (ESLint e ktlint)
    per imporre lo stile Airbnb~\cite{airbnb-javascript} per TypeScript e lo stile ufficiale~\cite{pinterest-ktlint} per Kotlin in tutta la codebase.
    Questo ha permesso di evitare bug comuni riconoscibili attraverso analisi statica
    e potenzialmente permette di rendere il codice, che è pubblico e open-source, maggiormente comprensibile per chi vorrà estenderlo in futuro.

    Inoltre, per garantire il corretto funzionamento delle componenti più cruciali, sono stati creati degli specifici unit test in grado di coprire il codice per quanto possibile.
    L'esecuzione dei test viene effettuata automaticamente da Travis CI su diverse piattaforma ad ogni operazione di push.

    Per quanto riguarda il backend è stato utilizzato JUnit 5 con l'ausilio delle estensioni di Vert.x.
    I report sulla copertura vengono raccolti tramite JaCoCo.

    \unsure[inline]{Quali dettagli dovrei fornire dei test del backend?}

    Anche il frontend è stato testato tramite unit testing.
    In particolare, si è utilizzato la suite Jest per l'esecuzione dei test e la raccolta dei dati di copertura.

    \unsure[inline]{Quali dettagli dovrei fornire dei test del frontend?}

  \section{Valutazione dell'interfaccia}
    % TODO
    \todo[inline]{Qui andrò a inserire i report di Lighthouse}

    % TODO inserisci lighthouse

  \section{Performance}
    % TODO
    \unsure[inline]{Come valuto i risultati? Misure di Performance?}
