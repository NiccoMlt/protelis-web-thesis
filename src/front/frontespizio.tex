\begin{Preambolo*}
  \usepackage{fontspec}
  \defaultfontfeatures{ Scale = MatchUppercase }
  \setmainfont{libertinusserif}[
    Scale=1.0,
    Ligatures={Common, TeX},
    % Numbers={OldStyle, Proportional},
    UprightFont={*-regular},
    BoldFont={*-bold},
    ItalicFont={*-italic},
    BoldItalicFont={*-bolditalic},
    Extension=.otf
  ]
  \setsansfont{libertinussans}[
    Ligatures={Common, TeX},
    % Numbers={OldStyle, Proportional},
    UprightFont={*-regular},
    BoldFont={*-bold},
    ItalicFont={*-italic},
    % BoldItalicFont={*-bolditalic},
    Extension=.otf
  ]
  \setmonofont{libertinusmono}[
    Scale=0.95,
    UprightFont={*-regular},
    % BoldFont={*-bold},
    % ItalicFont={*-italic},
    % BoldItalicFont={*-bolditalic},
    Extension=.otf
  ]
\end{Preambolo*}
\begin{frontespizio}
  \Universita{Bologna}        % aggiunge da sé “Università degli Studi di”.
  \Istituzione{%
    Alma Mater Studiorum --- Università di Bologna \\%
    Campus di Cesena%
  }
  \Divisione{Dipartimento di Informatica --- Scienza e Ingegneria}
  \Corso[Laurea magistrale]{Ingegneria e Scienze Informatiche}
  \Annoaccademico{2019--2020}
  \Titolo{Progettazione di una piattaforma web\\per la simulazione di programmi aggregati}
  \Sottotitolo{Tesi in Pervasive Computing}
  % \Preambolo{\renewcommand{\frontsmallfont}[1]{\small}}       % non viene stampata la matricola
  % \Preambolo{\renewcommand{\frontsmallfont}[1]{\small Matr.}} % abbrevia la matricola
  \Candidato[840825]{Niccolò~Maltoni}
  \NCandidato{Presentata da}  % sostituisce la parola “Candidato”
  \Relatore{Prof.~Mirko~Viroli}
  \Correlatore{Prof.~Danilo~Pianini}
  \Piede{%                    % sostituisce la scritta “Anno Accademico” nel piede
    III sessione di laurea \\%
    Anno Accademico 2019--2020%
  }
\end{frontespizio}

% Necessario per Overleaf: compila il TeX del frontespizio subito dopo averlo generato
\IfFileExists{\jobname-frn.pdf}{}{%
\immediate\write18{lualatex \jobname-frn}}
