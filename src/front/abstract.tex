\begin{abstract}

  % Lo sviluppo tecnologico avuto nei sistemi informatici nel corso degli ultimi anni è stato notevole;
  % questo ha portato alla necessità di programmare sistemi distribuiti composti da numerosi dispositivi, possibilmente eterogenei,
  % che devono potersi coordinare tra loro per poter portare a termine la computazione.
  % Data la complessità inerente al processo di sviluppo in questo contesto, si è sentita l'esigenza di indagare nuovi approcci da affiancare a quello tradizionale.
  % La \emph{programmazione aggregata} è uno di questi. Basandosi sull'impianto teorico del \emph{field calculus}, negli ultimi anni sono stati realizzati, da parte dell'Università di Bologna, linguaggi e framework
  % innovativi per la sua applicazione in contesti d'uso reale: \emph{Protelis} e \emph{ScaFi}.

  % La principale criticità legata a questo tipo di linguaggi, soprattutto in contesto didattico, riguarda la complessità di configurazione del sistema per l'esecuzione:
  % infatti la configurazione di una rete di dispositivi, necessaria per l'esecuzione del codice, può non essere banale, coinvolgendo il dispiegamento di un certo numero di dispositivi o la configurazione di un simulatore.

  La \emph{programmazione aggregata} è un approccio innovativo nato in tempi recenti per far fronte alla necessità di un punto di vista nuovo nella programmazione di sistemi distribuiti.
  In particolare, basandosi sull'impianto teorico del \emph{field calculus}, negli ultimi anni sono stati realizzati, da parte dell'Università di Bologna, linguaggi e framework
  innovativi per la sua applicazione in contesti d'uso reale: \emph{Protelis} e \emph{ScaFi}.

  La principale criticità che mina la diffusione di questo tipo di linguaggi è legata alla configurazione del sistema per l'esecuzione:
  è infatti necessario avere a disposizione una rete, reale o fisica, di dispositivi per l'esecuzione del codice e, soprattutto in contesto didattico,
  la necessità di dispiegare un certo numero di dispositivi o configurare un simulatore può costituire un ulteriore gradino di complessità.

  Lo scopo di questa tesi è progettare una piattaforma web che permetta di realizzare semplici programmi aggregati senza configurazione alcuna.
  È stato realizzato un sistema composto da un server esecutore, realizzato avvalendosi del simulatore Alchemist,
  e da un'applicazione web in React che permetta la scrittura del codice e il \emph{monitoring} dell'esecuzione.

% Entrambi si avvalgono della piattaforma JVM (\emph{\emph{J}ava \emph{V}irtual \emph{M}achine}) per poter essere eseguiti;
% come vedremo, questo garantisce numerose proprietà, ma può essere limitante in contesti didattici.
% Infatti, la necessità di possedere una rete reale di dispositivi o di configurare un simulatore per l'esecuzione
% aggiunge ulteriore complessità per un novizio che voglia approcciarsi alla tecnologia.
% Inoltre, non è da ignorare nemmeno la necessità di configurazione di un progetto Gradle o SBT completo per poter realizzare un prototipo minimale.

% l'evoluzione tecnologica ha permesso di realizzare applicazioni utilizzabili tramite browser con livelli di complessità comparabili alle controparti desktop,
% senza il carico aggiuntivo, dal punto di vista dell'utente, dell'installazione e della configurazione.
% Inoltre, servizi complessi possono non dipendere esclusivamente dalle risorse computazionali dei dispositivi dell'utente,

% Si è dunque ritenuto utile realizzare un sistema web semplice e immediato da usare che permetta di abbozzare esempi di codice aggregato
% (Protelis, nel prototipo implementato per questa tesi) e poterlo eseguire senza disporre di una rete di dispositivi o di un simulatore configurati per lo scopo.

\end{abstract}
