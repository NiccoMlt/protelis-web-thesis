\chapter{Progettazione di sistemi web}\label{chap:web}
  Il World Wide Web ha assunto un ruolo sempre più centrale nella quotidianità delle persone e nelle dinamiche di business.
  In particolare, il modello di comunicazione è sempre più virato verso scenari distribuiti,
  nei quali piattaforme eterogenee riescono a comunicare tra loro condividendo informazioni di diverso tipo attraverso la rete internet.

  Anche i pattern di progettazione e le tecnologie implementative sono cresciute altrettanto velocemente negli ultimi anni, cambiando anche radicalmente gli approcci di interazione possibili.
  Risulta dunque importante prestare attenzione allo stato dell'arte in tal senso, chiarendo quali siano i pattern più adatti e moderni per il contesto d'uso di questa tesi.

  \section{Architetture, framework e stack}\label{sec:web-architecture}

  Con \emph{sistema web} si intende genericamente un sistema software distribuito che coinvolge una o più entità server che espongono in rete API di varia natura, con le quali entità client possono comunicare per usufruire dei servizi.
  Generalmente, in contesto web i client sono costituiti da pagine web aperte nei browser degli utenti.

  Le possibilità di progettazione di un'applicazione web possono essere molto differenti e nel tempo si è vista una vera e propria evoluzione in tal senso:

  \begin{enumerate}
    \item
      nel periodo iniziale del web, ciascuna pagina web era inviata al client come un documento statico;
      l'interazione dell'utente con il sistema avveniva attraverso la navigazione, che comportava l'apertura di una sequenza di pagine a seconda delle esigenze.
      Questo tipo di interazione era però lenta, in quanto coinvolgeva sempre la ricezione di una nuova pagina dal server.
    \item
      successivamente, nel 1995, con l'introduzione da parte di NetScape di JavaScript (di cui tratteremo meglio nella~\Cref{subsec:js}) come linguaggio di scripting client-side, i programmatori hanno avuto la possibilità di inserire elementi dinamici nelle pagine web.
      In questo modo, era possibile effettuare alcune operazioni anche localmente, riducendo di fatto il numero di pagine intere scambiate con il server.
    \item
      un'ulteriore innovazione apparve l'anno seguente, quando Macromedia introdusse \emph{Flash}:
      esso era un plugin per i browser che permetteva di riprodurre animazioni vettoriali e gestire le interazioni con l'utente, in modo simile a quanto fatto da JavaScript.
    \item
      il termine ``\emph{web application}'' nasce però con l'introduzione della versione 2.2 della Java Servlet Specification~\cite{java1999specification} nel 1999.
      anche in questo caso, però, il server ha un ruolo centrale e ancora il concetto di \emph{ajax} (\emph{Asynchronous JavaScript + XML}) non è stato introdotto.
    \item
      successivamente vi sono stati diversi miglioramenti incrementali, fino ad arrivare allo standard HTML5~\cite{Smith2008}:
      con quest'ultimo, infatti, introduce il supporto nativo ai contenuti multimediali ed arricchisce la semantica del documento, oltre a migliorare l'integrazione con JS\@.
      Con questo standard, diventa sempre più comune il concetto di \emph{Single-page application} (SPA), secondo il quale la pagina viene caricata una sola volta, e poi modificata dinamicamente tramite chiamate specifiche al server.
      Nascono numerosi framework client-side (come Angular, Ember o React) e l'impiego del server viene sempre più circoscritto al fornire API per accesso controllato ai dati (ad esempio, un database) o per computazioni complesse.
  \end{enumerate}

  % TODO: Service-Oriented Architecture e Resource-Oriented architecture

  \section{Linguaggi ad uso web}\label{sec:lang}
    In questa \nameCref{sec:lang} verranno analizzati i principali linguaggi di programmazione utilizzati recentemente per lo sviluppo della componente frontend delle applicazioni web.
    In particolare, verranno presi in considerazione i due linguaggi più popolari, ossia \emph{JavaScript} e il suo superset \emph{TypeScript}, e due linguaggi di nicchia che offrono una valida alternativa: \emph{Scala.js} e \emph{Kotlin/JS}\@.

    \subsection{JavaScript e ECMAScript}\label{subsec:js}

    \emph{JavaScript} è un linguaggio di scripting debolmente tipizzato orientato agli oggetti e agli eventi.
    Sviluppato originariamente nel 1995 da Brendan Eich della Netscape Communications (inizialmente con il nome di \emph{Mocha} e poi \emph{LiveScript}),
    esso è stato concepito con lo scopo di avere un linguaggio di scripting per il browser Netscape Navigator più semplice da apprendere rispetto a quelli esistenti.
    JavaScript è stato standardizzato per la prima volta nel 1997 dalla ECMA con il nome di \emph{ECMAScript}~\cite{ECMA-262,ISOCS1998} e l'attuale versione è la decima.

    Il linguaggio è attualmente il più popolare per uso web, in quanto l'unico ad essere supportato da tutti i browser moderni, almeno nelle sue feature principali.
    Diversi linguaggi, come quelli che vedremo in seguito, vengono transpilati in una versione sufficientemente supportata di JS per poter essere eseguiti nei browser.

    Analizzato dal punto di vista tecnico, esso presenta i seguenti aspetti strutturali:

    \begin{description}
      \item[Prototype-based] % TODO
      % TODO
    \end{description}

    \subsection{TypeScript}\label{subsec:ts}
    \subsection{Scala.js}
    \subsection{Kotlin/Multiplatform e Kotlin/JS}
