\addchap{Introduzione}

% Nel corso degli ultimi anni i sistemi informatici hanno avuto uno sviluppo sempre crescente,
Nel corso degli ultimi anni i sistemi informatici hanno avuto uno sviluppo notevole.
Grazie all'incremento della potenza computazionale disponibile a basso costo, alla riduzione delle dimensioni delle unità di calcolo e alla diffusione delle reti wireless,
ormai gli ambienti in cui vive la maggior parte della popolazione sono pervasi di sensori (come è il caso dell'IoT) e di dispositivi ``smart'' (come smartphone e tablet).

Questo ha portato alla necessità di programmare sistemi distribuiti composti da numerosi dispositivi che devono potersi coordinare tra loro
per poter portare a termine la computazione.

La \emph{programmazione aggregata} è un approccio promettente per lo sviluppo di sistemi di questo tipo.
Tale paradigma è basato sull'impianto teorico del \emph{field calculus} e ha visto negli ultimi anni la realizzazione,
da parte dell'Università di Bologna, di linguaggi e framework innovativi per la sua applicazione in contesti d'uso reale:
è questo il caso di \emph{Protelis} e \emph{ScaFi}.

Entrambi si avvalgono della piattaforma JVM per poter essere eseguiti;
come vedremo, questo garantisce numerose proprietà, ma può essere limitante in contesti didattici.
Infatti, la necessità di configurare un progetto Gradle o SBT completo per poter realizzare un prototipo minimale
aggiunge ulteriore complessità per un novizio che voglia approcciarsi alla tecnologia.
Inoltre, non è da ignorare nemmeno la configurazione di una rete reale o simulata di dispositivi per l'esecuzione.

In tempi recenti anche il web è maturato molto:
l'evoluzione tecnologica ha permesso di realizzare applicazione accessibili tramite browser con livelli di complessità comparabili alle controparti desktop,
senza il carico aggiuntivo, dal punto di vista dell'utente, dell'installazione e della configurazione.
Inoltre, servizi complessi possono non dipendere esclusivamente dalle risorse computazionali dei dispositivi dell'utente,
bensì sfruttarle se e quando necessario.

Si è dunque ritenuto utile realizzare un sistema web semplice e immediato da usare che permetta di abbozzare esempi di codice aggregato
(Protelis, nel prototipo implementato per questa tesi) e poterlo eseguire senza disporre di una rete di dispositivi o di un simulatore configurati per lo scopo.
In particolare, si è deciso di propendere per l'implementazione di un backend reattivo su piattaforma JVM che esegue, su una rete di dispositivi simulata tramite Alchemist, il codice inviato da un client.
Il client è una \emph{Single-Page Application} statica che permette lo sviluppo di codice direttamente dal browser e comunica con il server tramite bus di eventi.

% Nel corso degli ultimi anni i sistemi informatici hanno avuto uno sviluppo
% incrementale, non solo incrementando la potenza computazionale, ma anche
% riducendo drasticamente le dimensioni delle unità di calcolo. Questo ha portato
% alla nascita di nuovi dispositivi indossabili (e portabili) come smartphone
% e smartwatch divenuti, per la maggior parte della popolazione mondiale, be-
% ni di prima necessità. La miniaturizzazione delle macchine computazionali,
% inoltre, ha portato alla creazione di sensori intelligenti che possono essere
% installati nei posti più disparati. Quando quest'enorme insieme di dispositivi
% si connette in una rete comune come Internet, allora si inizia a parlare di
% Internet of Thing (IoT). In poco tempo si è passati da quello che è stato
% il paradigma di computazione standard per decine di anni, cioè il desktop
% computing (la computazione avviene in un singolo dispositivo), a pervasive
% computing (la computazione non è allocata in un solo dispositivo ma viene
% eseguito da un insieme di dispositivi che riescono a portare a termine un
% risultato comune). I linguaggi e i paradigmi di programmazione standard non
% sono riusciti a garantire delle tecniche di programmazione atte a gestire tali
% sistemi, perciò negli anni si è provato ad approcciarsi al problema in diversi
% modi. Le tecniche di Aggregate Programming (basate sul concetto di field
% calculus), ad esempio, aggiungono un livello d'astrazione tale da nascondere
% quelli che sono i dettagli legati alla rete (tipologia di connessione, topologia..)
% descrivendo l'algoritmo da eseguire su un insieme di dispositivi in termini
% del loro comportamento aggregato.
% [...]
% Il comportamento di un programma aggregato, vuoi per la nuova tipologia
% di pensiero, vuoi per la sua complessità, non è sempre facile da definire
% e perciò, per verificare il comportamento di questi sistemi, è nata l'esigenza
% di poter simulare sistemi reali dove avviare programmi aggregati. In questo
% modo si ha un risultato visivo della simulazione e si verifica la correttezza
% della logica del programma. Data questa esigenza, l'obiettivo di questa tesi
% di laurea è quello di sviluppare un front-end di simulazioni aggregate de-
% scritte in ScaFi per permettere di testare e visualizzare scenari di utilizzo di
% tale framework. Per front-end non si intende solamente un semplice sistema
% che renderizzi dei dati prodotti dalle simulazioni aggregate ma, che inoltre
% aggiunga l'interazione con il sistema aggregato, la configurazione e l'avvio
% di simulazione.

% Al giorno d'oggi la diffusione di dispositivi mobili sempre più avanzati
% e l'evoluzione delle tecnologie wireless ha cambiato radicalmente la
% comunicazione e la vita delle persone. Qualsiasi utente in possesso di
% uno smartphone o un tablet è ora in grado di ricevere informazioni e
% il numero di servizi sta aumentando in maniera esponenziale.
% Questo ha permesso agli sviluppatori software di investigare nel
% campo del cosiddetto pervasive computing: ad esempio l'applicazione di Star-
% bucks rileva la presenza della propria caffetteria preferita nelle
% vicinanze e propone all'utente sullo schermo la tessera fedeltà per pagare
% direttamente con lo smartphone.

% La diffusione di dispositivi mobili sempre più avanzati e la parallela evoluzione
% delle tecnologie wireless, ha rivoluzionato sempre più il mondo della
% comunicazione e lo stile di vita delle persone, permettendo negli ultimi anni lo
% sviluppo di un elevato numero di servizi strettamente legati alla mobilità degli
% utenti al fine di migliorare la loro esperienza nelle esigenze quotidiane. La
% possibilità di avere a bordo del proprio smartphone una serie di componenti
% hardware sempre più precisi e potenti, ha allargato l'orizzonte di molti
% sviluppatori software soprattutto in ambito pervasive-computing
% [...]
% Per tale scopo si parte da un'infrastruttura base sviluppata in ambito accademico
% e i cui concetti derivano dal campo dello Spatial Computing: ogni nodo del
% sistema non è direttamente a conoscenza di tutti gli altri presenti, ma interagisce
% solo coi propri vicini in modo da costruire una mappa che tenga traccia di tutti
% i nodi presenti e relative distanze da sé stesso, allo scopo di auto-organizzarsi
% per svolgere un certo task applicativo. Questa primo prototipo dell'infrastruttura
% sfrutta la nozione di “vicinato” sulla base delle distanze reali

% I sistemi computazionali moderni sono sempre più complessi, in
% particolare si sta assistendo all'esplosione dei sistemi pervasivi, cioè sistemi
% computazionali distribuiti su larga scala che comprendono al loro interno un
% elevato numero di componenti e dispositivi eterogenei fra loro. Dal punto di
% vista ingegneristico è quindi importante assicurarsi che questi sistemi abbiano
% diverse proprietà come scalabilità, reattività, tolleranza ai guasti unite a buone
% performance. Realizzare queste proprietà tramite un coordinamento centrale è
% praticamente impossibile su sistemi grandi come i sistemi pervasivi, sorge
% quindi la necessità di un nuovo approccio che le garantisca, questo approccio
% prende il nome di auto-organizzazione.

% Negli ultimi anni il Web ha assunto un ruolo, di giorno in giorno, sempre più
% centrale. La presenza sempre più diffusa e disponibile di una connessione
% ad Internet ad alta velocità, ha portato il Web ovunque, anche in luoghi
% dove, pochi anni fa, non si sarebbe mai immaginato.
% La realizzazione e la diffusione, in continua crescita, di applicazioni Web
% con interfacce grafiche sempre più accattivanti e funzionalità sempre più
% avanzate hanno portato ad un utilizzo del Web sempre più pervasivo, sia
% all'interno degli ambienti lavorativi che all'interno delle mura domestiche.

% Il Web nel corso della sua esistenza ha subito un mutamento dovuto in parte
% dalle richieste del mercato, ma soprattutto dall'evoluzione e la nascita co-
% stante delle numerose tecnologie coinvolte in esso. Si è passati da un'iniziale
% semplice diffusione di contenuti statici, ad una successiva collezione di siti
% web, dapprima con limitate presenze di dinamicità e interattività (a causa
% dei limiti tecnologici), ma successivamente poi evoluti alle attuali applicazioni
% web moderne che hanno colmato il gap con le applicazioni desktop, sia a
% livello tecnologico, che a livello di diffusione effettiva sul mercato.
% Tali applicazioni web moderne possono presentare un grado di complessità
% paragonabile in tutto e per tutto ai sistemi software desktop tradizionali;

Il documento è suddiviso in tre \nameCrefs{part:background} principali.

Nella~\Cref{part:background} viene fatta una disamina del contesto nel quale l'elaborato di tesi va a inserirsi.
In particolare, nel~\Cref{chap:motivations} vengono esaminate più nel dettaglio le ragione per le quali il progetto è stato realizzato,
con particolare attenzione allo stato dell'arte e a possibili soluzioni a problemi simili.
Nel~\Cref{chap:aggregate} del documento viene fatta un'introduzione alla programmazione aggregata, con particolare attenzione all'impianto teorico su cui si basa e ai linguaggi ad essa collegati.
Nel~\Cref{chap:web} viene invece fatta riportato il risultato della fase di studio dello stato dell'arte in merito allo sviluppo di sistemi web,
con particolare attenzione ai linguaggi che permettono la realizzazione di applicazioni frontend in esecuzione sui browser degli utenti.

Nella~\Cref{part:contribution} viene analizzato il processo di progettazione del sistema a cui questo elaborato di tesi si riferisce e di sviluppo del prototipo.
Ciascuna fase del processo è descritta in un \nameCref{chap:requirements} dedicato.

Infine, nella~\Cref{part:conclusion} vengono raccolti i risultati del processo di sviluppo (\Cref{chap:evaluation}), presentate le prospettive future (\Cref{chap:future}) ed enunciate brevemente le considerazioni finali (\Cref{chap:considerations}).
