\chapter{Considerazioni finali}\label{ch:considerations}
  L'obiettivo della tesi era quello di progettare un sistema che permettesse, senza particolari configurazioni,
  di scrivere codice aggregato ed eseguirlo su una rete di esempio.
  Esso doveva essere facilmente accessibile e immediatamente utilizzabile, dunque si è scelto di indirizzarsi verso tecnologie web.

  Per la realizzazione del prototipo, ci si è focalizzati su un solo linguaggio di programmazione aggregata: Protelis.
  Inoltre, per motivi di realizzabilità concreta, si è deciso di simulare la rete di dispositivi su cui eseguire il codice,
  anziché utilizzarne una fisica.

  La realizzazione del sistema nel suo complesso ha comportato la progettazione di due applicazioni su differenti piattaforme e con diverse tecnologie e linguaggi:
  la componente server è stata realizzata avvalendosi del framework Vert.x nel linguaggio Kotlin,
  mentre l'interfaccia web è stata implementata come Single-Page Application avvalendosi di React in TypeScript.

  Per la soddisfazione dei requisiti è stato dunque richiesto uno studio approfondito, anche a causa della profonda diversità di questi due sistemi.
  La decisione di appoggiarsi a soluzioni per la programmazione web moderne è stata però fondamentale per ottenere un sistema davvero di uso immediato come richiesto.

  Nonostante il sistema realizzato sia correttamente funzionante e soddisfi i requisiti, esso è solo un prototipo e, soprattutto con l'impiego di piattaforme economiche come quelle fornite con il piano \emph{Free} di Heroku, non sarebbe in grado di reggere il carico di un elevato numero di utenti.
  L'attenzione dedicata in fase di progettazione, però, lascia aperte molte strade per l'implementazione in modo semplice di soluzioni maggiormente adatte a casi d'uso complessi.

  Si ritiene dunque che il lavoro svolto abbia raggiunto gli obiettivi prefissati, diventando un'occasione formativa notevole e portando alla realizzazione di un sistema potenzialmente molto utile per la divulgazione della programmazione aggregata.
